\documentclass[12pt,titlepage]{article}
\usepackage{mathtools}
\usepackage[italian]{babel}
\usepackage{graphicx}
\usepackage[utf8]{inputenc}
\usepackage{float}
\usepackage{tabularx}
\usepackage{chngpage}
\usepackage[toc,page]{appendix}
\usepackage{gensymb}
\usepackage{subcaption}
\usepackage{tikz}
\usepackage{circuitikz}

\makeatletter
\def\pgf@circ@myvoltmeter@path#1{\pgf@circ@bipole@path{myvoltmeter}{#1}}
\tikzset{myvoltmeter/.style = {\circuitikzbasekey, /tikz/to
                               path=\pgf@circ@myvoltmeter@path}}
\pgfcircdeclarebipole{}{\ctikzvalof{bipoles/voltmeter/height}}{myvoltmeter}{\ctikzvalof{bipoles/voltmeter/height}}{\ctikzvalof{bipoles/voltmeter/width}}{
    \def\pgf@circ@temp{right}
    \ifx\tikz@res@label@pos\pgf@circ@temp
        \pgf@circ@res@step=-1.2\pgf@circ@res@up
    \else
        \def\pgf@circ@temp{below}
        \ifx\tikz@res@label@pos\pgf@circ@temp
            \pgf@circ@res@step=-1.2\pgf@circ@res@up
        \else
            \pgf@circ@res@step=1.2\pgf@circ@res@up
        \fi
    \fi

    \pgfpathmoveto{\pgfpoint{\pgf@circ@res@left}{\pgf@circ@res@zero}}       
    \pgfpointorigin \pgf@circ@res@other =  \pgf@x  \advance \pgf@circ@res@other by -\pgf@circ@res@up
    \pgfpathlineto{\pgfpoint{\pgf@circ@res@other}{\pgf@circ@res@zero}}
    \pgfusepath{draw}

    \pgfsetlinewidth{\pgfkeysvalueof{/tikz/circuitikz/bipoles/thickness}\pgfstartlinewidth}

        \pgfscope
            \pgfpathcircle{\pgfpointorigin}{.9\pgf@circ@res@up} % change this if you want to touch the wires
            \pgfusepath{draw}       
        \endpgfscope    

    \pgfsetlinewidth{\pgfstartlinewidth}
    \pgftransformrotate{90}
    \pgfsetarrowsend{latex}
    \pgfpathmoveto{\pgfpoint{\pgf@circ@res@other}{\pgf@circ@res@down}}
    \pgfpathlineto{\pgfpoint{-\pgf@circ@res@other}{\pgf@circ@res@up}} % change this if you want to touch the wires
    \pgfusepath{draw} % comment this if you don't need the diagonal arrow
    \pgfsetarrowsend{}


    \pgfpathmoveto{\pgfpoint{-\pgf@circ@res@other}{\pgf@circ@res@zero}}
    \pgfpathlineto{\pgfpoint{\pgf@circ@res@right}{\pgf@circ@res@zero}}
    \pgfusepath{draw} % comment this if you don't need the diagonal arrow

    \pgfnode{circle}{center}{\textbf{V}}{}{}
}
\makeatother

\makeatletter
\def\pgf@circ@myl@path#1{\pgf@circ@bipole@path{myl}{#1}}
\tikzset{myl/.style = {\circuitikzbasekey, /tikz/to
                               path=\pgf@circ@myl@path}}
\pgfcircdeclarebipole{}{\ctikzvalof{bipoles/voltmeter/height}}{myl}{\ctikzvalof{bipoles/voltmeter/height}}{\ctikzvalof{bipoles/voltmeter/width}}{
    \def\pgf@circ@temp{right}
    \ifx\tikz@res@label@pos\pgf@circ@temp
        \pgf@circ@res@step=-1.2\pgf@circ@res@up
    \else
        \def\pgf@circ@temp{below}
        \ifx\tikz@res@label@pos\pgf@circ@temp
            \pgf@circ@res@step=-1.2\pgf@circ@res@up
        \else
            \pgf@circ@res@step=1.2\pgf@circ@res@up
        \fi
    \fi

    \pgfpathmoveto{\pgfpoint{\pgf@circ@res@left}{\pgf@circ@res@zero}}       
    \pgfpointorigin \pgf@circ@res@other =  \pgf@x  \advance \pgf@circ@res@other by -\pgf@circ@res@up
    \pgfpathlineto{\pgfpoint{\pgf@circ@res@other}{\pgf@circ@res@zero}}
    \pgfusepath{draw}

    \pgfsetlinewidth{\pgfkeysvalueof{/tikz/circuitikz/bipoles/thickness}\pgfstartlinewidth}

        \pgfscope
            \pgfpathcircle{\pgfpointorigin}{.9\pgf@circ@res@up} % change this if you want to touch the wires
            \pgfusepath{draw}       
        \endpgfscope    

    \pgfsetlinewidth{\pgfstartlinewidth}
    \pgftransformrotate{90}
    \pgfsetarrowsend{latex}
    \pgfpathmoveto{\pgfpoint{\pgf@circ@res@other}{\pgf@circ@res@down}}
    \pgfpathlineto{\pgfpoint{-\pgf@circ@res@other}{\pgf@circ@res@up}} % change this if you want to touch the wires
    \pgfusepath{draw} % comment this if you don't need the diagonal arrow
    \pgfsetarrowsend{}


    \pgfpathmoveto{\pgfpoint{-\pgf@circ@res@other}{\pgf@circ@res@zero}}
    \pgfpathlineto{\pgfpoint{\pgf@circ@res@right}{\pgf@circ@res@zero}}
    \pgfusepath{draw} % comment this if you don't need the diagonal arrow

    \pgfnode{circle}{center}{\textbf{}}{}{}
}
\makeatother

\addtolength{\topmargin}{-.875in}
\textheight=674pt

\title{\textbf{MISURE VOLT- AMPEROMETRICHE} }
\author{Canteri Marco\\Biasi Lorenzo\\Damiani Emily}
\date{}

\begin{document}
	\maketitle
	\tableofcontents
	\renewcommand{\abstractname}{Abstract}
	
	\begin{abstract}
In questa esperienza misureremo il valore di una resistenza analizzando con un tester ICE le differenze di potenziale ai suoi capi e la corrente che gli passa attraverso. Verificheremo quindi il comportamento ohmico di tale resistenza. Lo facciamo in due modi, utilizzando un circuito con amperometro a monte e un circuito con amperometro a valle. Inoltre utilizzeremo due fondo scala diversi per ciascun circuito. 
	\end{abstract}

\section{Obiettivi}
\begin{itemize}
\item Determinare il valore di resistenze tramite misure volt-amperometriche con il tester
ICE, in configurazione amperometro “a monte” e “a valle”.
\item Verificare il comportamento ohmico della resistenza e confrontare i valori ottenuti e
le loro incertezze tra di loro e con il valore ottenuto da un multimetro digitale.
\end{itemize}
\section{Strumenti}
\begin{itemize}
\item Due tester ICE (tolleranza 5\% fondoscala)
\item resistenza da misurare
\item generatore di tensione variabile
\item breadboard
\item cavi di collegamento
\end{itemize}
\section{Procedura di misura}
Abbiamo eseguito 4 misure di resistenza in totale. Le prime due misure sono state eseguite con un circuito con amperomentro a monte rappresentato in figura \eqref{monte}, le altre due sono state eseguite con amperometro a valle in figura \eqref{valle}. Entrambe le coppie con 2 fondo scale diverse. Abbiamo effettuato la misura di differenza di potenziale e corrente con due tester ICE. L'incertezza di tali misure sono dovute esclusivamente alla risoluzione $\Delta X$ della scala del tester: 5\% del fondo scala. Quindi l'incertezza standard è:
\[ \sigma X = \frac{\Delta X}{\sqrt{12}}\]
In realtà l'incertezza sulla misura è data anche dalle caratteristiche non ideali dei tester utilizzati. Nella seconda parte abbiamo analizzato i circuiti del tester utilizzato e introdotto alcuni elementi di correzione sulle misure effettuate per avere il valore più corretto della resistenza da misurare.
\begin{figure}[H]
    \centering
    \begin{subfigure}[b]{0.3\textwidth}
        \begin{circuitikz}
      \draw (0,0) 
      to[battery](0,2)
      to[ammeter](2,2)
      to[myvoltmeter](2,0)
      to[short](0,0)
      ;
      \draw(2,2)
      to[short](3,2)
      to[R=$R_x$](3,0)
      to[short](2,0);
      \end{circuitikz}  
        \caption{Circuito amperometro a monte}
        \label{monte}
    \end{subfigure}
    \qquad \qquad%add desired spacing between images, e. g. ~, \quad, \qquad, \hfill etc. 
      %(or a blank line to force the subfigure onto a new line)
    \begin{subfigure}[b]{0.3\textwidth}
       \begin{circuitikz}
      \draw (0,0) 
      to[battery](0,2)
      to[short](1,2)
      to[myvoltmeter](1,0)
      to[short](0,0)
      ;
      \draw(1,2)
      to[ammeter](3,2)
      to[R=$R_x$](3,0)
      to[short](1,0);
      \end{circuitikz}  
        \caption{Circuito amperometro a valle}
        \label{valle}
    \end{subfigure}
   
\end{figure}

\section{Circuito amperometro a monte}
\section{Cirucito amperometro a valle}
\section{Analisi circuito tester ICE}
\begin{figure}[h!]
  \centering
    \begin{circuitikz}
      \draw (0,0) 
       node[label={below:A}] {}
       node[yshift=-1ex] {=}
       to[short,o-*] (0,2)
       to[short] (0,4)
       to[short] (5.5,4)
       to[ammeter]node[right = -0.5em,yshift=-2.3ex] {$R_A = 1600\ohm$} node[right = -7.5em,yshift=-2.3ex] {$I_{fs} = 40\mu A$} 
       node[right = 0.1em,yshift=2.3ex] {$+$} node[right =-3.5em,yshift=2.3ex] {$-$}(6.5,4)
       to[short] (12,4)
       to[short,-*] (12,2)
       to[R= $720\ohm$,-o](12,0)node[right = 0.1em] {$A_6$}
       node[label={below:$50\mu A$}] {};
      \draw (0,2)
      to[R=$R_1$,*-*]node[below,right=-3.6em,yshift=-2.3ex] {0.064\ohm}(2,2)
      to[R=$R_2$,*-*]node[below,right=-3.6em,yshift=-2.3ex] {0.576\ohm}(4,2)
      to[R=$R_3$,*-*]node[below,right=-3.4em,yshift=-2.3ex] {5.76\ohm}(6,2)
      to[R=$R_4$,*-*]node[below,right=-3.4em,yshift=-2.3ex] {57.6\ohm}(8,2)
      to[R=$R_5$,*-*]node[below,right=-3.4em,yshift=-2.3ex] {576\ohm}(10,2)
      to[R=$R_6$,*-*]node[below,right=-3.6em,yshift=-2.3ex] {5760\ohm}(12,2);
      \draw (2,2)
      to[short,-o](2,0)node[right = 0.1em] {$A_1$}
      node[label={below:5A}] {};
      \draw (4,2)
      to[short,-o](4,0)node[right = 0.1em] {$A_2$}
      node[label={below:500mA}] {};
      \draw (6,2)
      to[short,-o](6,0)node[right = 0.1em] {$A_3$}
      node[label={below:50mA}] {};
      \draw (8,2)
      to[short,-o](8,0)node[right = 0.1em] {$A_4$}
      node[label={below:5mA}] {};
      \draw (10,2)
      to[short,-o](10,0)node[right = 0.1em] {$A_5$}
      node[label={below:$500\mu A$}] {};
    \end{circuitikz}
    \caption{Schema semplificato tester ICE amperometro}
    \label{ICE_amp}
\end{figure}
Il circuito in figura \eqref{ICE_amp} una volta collegato all'interno del circuito tramite la bocchetta $A_i$ può essere schemattizato come segue
\begin{figure}[H]
  \centering
    \begin{circuitikz}
      \draw (0,0) 
      %node[label={below:$A_i$}] {}
      node[xshift=-1.5ex] {$A_i$}
      to[short,o-*](2,0)
      to[R=$R_{sh}$,-*](2,-2)
      to[short,-o]node[xshift=-4.0ex] {A}(0,-2);
      \draw(2,0)
      to[R=$R_S$](4,0)
      to[ammeter](6,0)
      to[R=$R_A$](6,-2)
      to[short,-*](2,-2);
      \end{circuitikz}  
\end{figure}
\section{Conclusioni}
\end{document}
