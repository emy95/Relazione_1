\documentclass[12pt]{article}
\usepackage{mathtools}
\usepackage[italian]{babel}
\usepackage{graphicx}
\usepackage[utf8]{inputenc}
\usepackage{float}
\usepackage{tabularx}
\usepackage{chngpage}
\usepackage[toc,page]{appendix}
\usepackage{gensymb}
\usepackage{subcaption}
\usepackage{tikz}
\usepackage{circuitikz}

\input{volt.tex}



\addtolength{\topmargin}{-.875in}
\textheight=674pt

\begin{document}
  \input{titolo.tex}
	\newpage
	\renewcommand{\abstractname}{Abstract}
	
	\begin{abstract}
In questa esperienza misureremo il valore di una resistenza analizzando con un tester ICE le differenze di potenziale ai suoi capi e la corrente che gli passa attraverso. Verificheremo quindi il comportamento ohmico di tale resistenza. Lo facciamo in due modi, utilizzando un circuito con amperometro a monte e un circuito con amperometro a valle. Inoltre utilizzeremo due fondo scala diversi per ciascun circuito. 
	\end{abstract}
  \vspace{10em}
\tableofcontents
  \newpage
\section{Obiettivi}
\begin{itemize}
\item Determinare il valore di resistenze tramite misure volt-amperometriche con il tester
ICE, in configurazione amperometro “a monte” e “a valle”.
\item Verificare il comportamento ohmico della resistenza e confrontare i valori ottenuti e
le loro incertezze tra di loro e con il valore ottenuto da un multimetro digitale.
\end{itemize}
\section{Strumenti}
\begin{itemize}
\item Due tester ICE (tolleranza 5\% fondoscala)
\item resistenza da misurare
\item generatore di tensione variabile
\item breadboard
\item cavi di collegamento
\end{itemize}
\section{Procedura di misura}
Abbiamo eseguito 4 misure di resistenza in totale. Le prime due misure sono state eseguite con un circuito con amperomentro a monte rappresentato in figura \eqref{monte}, le altre due sono state eseguite con amperometro a valle in figura \eqref{valle}. Entrambe le coppie con 2 fondo scale diverse. Abbiamo effettuato la misura di differenza di potenziale e corrente con due tester ICE. L'incertezza di tali misure sono dovute esclusivamente alla risoluzione $\Delta X$ della scala del tester: 5\% del fondo scala. Quindi l'incertezza standard è:
\[ \sigma X = \frac{\Delta X}{\sqrt{12}}\]
In realtà l'incertezza sulla misura è data anche dalle caratteristiche non ideali dei tester utilizzati. Nella seconda parte abbiamo analizzato i circuiti del tester utilizzato e introdotto alcuni elementi di correzione sulle misure effettuate per avere il valore più corretto della resistenza da misurare.
\begin{figure}[H]
    \centering
    \begin{subfigure}[b]{0.3\textwidth}
        \begin{circuitikz}
      \draw (0,0) 
      to[battery](0,2)
      to[ammeter](2,2)
      to[myvoltmeter](2,0)
      to[short](0,0)
      ;
      \draw(2,2)
      to[short](3,2)
      to[R=$R_x$](3,0)
      to[short](2,0);
      \end{circuitikz}  
        \caption{Circuito amperometro a monte}
        \label{monte}
    \end{subfigure}
    \qquad \qquad%add desired spacing between images, e. g. ~, \quad, \qquad, \hfill etc. 
      %(or a blank line to force the subfigure onto a new line)
    \begin{subfigure}[b]{0.3\textwidth}
       \begin{circuitikz}
      \draw (0,0) 
      to[battery](0,2)
      to[short](1,2)
      to[myvoltmeter](1,0)
      to[short](0,0)
      ;
      \draw(1,2)
      to[ammeter](3,2)
      to[R=$R_x$](3,0)
      to[short](1,0);
      \end{circuitikz}  
        \caption{Circuito amperometro a valle}
        \label{valle}
    \end{subfigure}
   \caption{Circuiti usati per la misura}
\end{figure}

\section{Circuito amperometro a monte}
\section{Cirucito amperometro a valle}
\section{Analisi circuito tester ICE}
\begin{figure}[h!]
  \centering
    \begin{circuitikz}
      \draw (0,0) 
       node[label={below:A}] {}
       node[yshift=-1ex] {=}
       to[short,o-*] (0,2)
       to[short] (0,4)
       to[short] (5.5,4)
       to[ammeter]node[right = -0.5em,yshift=-2.3ex] {$R_A = 1600\ohm$} node[right = -7.5em,yshift=-2.3ex] {$I_{fs} = 40\mu A$} 
       node[right = 0.1em,yshift=2.3ex] {$+$} node[right =-3.5em,yshift=2.3ex] {$-$}(6.5,4)
       to[short] (12,4)
       to[short,-*] (12,2)
       to[R= $720\ohm$,-o](12,0)node[right = 0.1em] {$A_6$}
       node[label={below:$50\mu A$}] {};
      \draw (0,2)
      to[R=$R_1$,*-*]node[below,right=-3.6em,yshift=-2.3ex] {0.064\ohm}(2,2)
      to[R=$R_2$,*-*]node[below,right=-3.6em,yshift=-2.3ex] {0.576\ohm}(4,2)
      to[R=$R_3$,*-*]node[below,right=-3.4em,yshift=-2.3ex] {5.76\ohm}(6,2)
      to[R=$R_4$,*-*]node[below,right=-3.4em,yshift=-2.3ex] {57.6\ohm}(8,2)
      to[R=$R_5$,*-*]node[below,right=-3.4em,yshift=-2.3ex] {576\ohm}(10,2)
      to[R=$R_6$,*-*]node[below,right=-3.6em,yshift=-2.3ex] {5760\ohm}(12,2);
      \draw (2,2)
      to[short,-o](2,0)node[right = 0.1em] {$A_1$}
      node[label={below:5A}] {};
      \draw (4,2)
      to[short,-o](4,0)node[right = 0.1em] {$A_2$}
      node[label={below:500mA}] {};
      \draw (6,2)
      to[short,-o](6,0)node[right = 0.1em] {$A_3$}
      node[label={below:50mA}] {};
      \draw (8,2)
      to[short,-o](8,0)node[right = 0.1em] {$A_4$}
      node[label={below:5mA}] {};
      \draw (10,2)
      to[short,-o](10,0)node[right = 0.1em] {$A_5$}
      node[label={below:$500\mu A$}] {};
    \end{circuitikz}
    \caption{Schema semplificato tester ICE amperometro}
    \label{ICE_amp}
\end{figure}
\begin{figure}[h!]
  \centering
    \begin{circuitikz}
      \draw (0,-1) 
       node[label={below:V}] {}
       node[yshift=-1ex] {=}
       to[short,o-*] (0,2)
       to[short] (0,4)
       to[short] (5.5,4)
       to[ammeter]node[right = -0.5em,yshift=-2.3ex] {$R_A = 1600\ohm$} node[right = -7.5em,yshift=-2.3ex] {$I_{fs} = 40\mu A$} 
       node[right = 0.1em,yshift=2.3ex] {$+$} node[right =-3.5em,yshift=2.3ex] {$-$}(6.5,4)
       to[short] (12,4)
       to[short,-*] (12,2)
       to[R= $720\ohm$,-o](14,2)node[right = 0.1em] {$B_7$}
       node[label={below:$100 mV$}] {};
       \draw (2,0)
      to[short,-o](2,-1)node[right = 0.1em] {$B_1$}
      node[label={below:2V}] {};
      \draw (4,0)
      to[short,-o](4,-1)node[right = 0.1em] {$B_2$}
      node[label={below:10V}] {};
      \draw (6,0)
      to[short,-o](6,-1)node[right = 0.1em] {$B_3$}
      node[label={below:50V}] {};
      \draw (8,0)
      to[short,-o](8,-1)node[right = 0.1em] {$B_4$}
      node[label={below:200V}] {};
      \draw (10,0)
      to[short,-o](10,-1)node[right = 0.1em] {$B_5$}
      node[label={below:500V}] {};
      \draw (12,0)
      to[short,-o](12,-1)node[right = 0.1em] {$B_6$}
      node[label={below:1000V}] {};
      \draw(12,2)
      to[short](12,1)
      to[R,l_=$6.52\,k\ohm$](8,1)
      to[R,l_=$32.2\,k\ohm$](5,1)
      to[short](2,1)
      to[short,-*](2,0) 
      to[R=$R_{V_1}$,*-*]node[below,right=-3.6em,yshift=-2.3ex] {160k\ohm}(4,0) 
      to[R=$R_{V_2}$,*-*]node[below,right=-3.6em,yshift=-2.3ex] {800k\ohm}(6,0) 
      to[R=$R_{V_3}$,*-*]node[below,right=-3.6em,yshift=-2.3ex] {3M\ohm}(8,0)  
      to[R=$R_{V_4}$,*-*]node[below,right=-3.6em,yshift=-2.3ex] {6M\ohm}(10,0) 
      to[R=$R_{V_5}$,*-*]node[below,right=-3.6em,yshift=-2.3ex] {10M\ohm}(12,0);
      \draw (0,2)
      to[R,l = $0.064\ohm$](2,2)
      to[R,l = $0.576\ohm$](4,2)
      to[R,l = $5.76\ohm$](6,2)
      to[R,l = $57.6\ohm$](8,2)
      to[R,l = $576\ohm$](10,2)
      to[R,l = $5760\ohm$](12,2);
    \end{circuitikz}
    \caption{Schema semplificato tester ICE voltmetro}
    \label{ICE_volt}
\end{figure}
\subsection{Analisi circuito amperometro a valle}
\begin{figure}[H]
  \centering
    \begin{circuitikz}
      \draw (0,0) 
      to[battery](0,6)
      to[short,-o]node[above,right=1.2em,yshift=1.5ex] {A}(3,6)
      to[short,i=$i$](3,5)
      to[short,i=$i_{sh}$](2,5)
      to[R=$R_{sh}$](2,2)
      to[short](3,2)
      to[R=$R_x$,i>^=$i$,-o]node[below,yshift=-1.3ex]{B}(3,0)
      to[short](0,0);
      \draw (3,5)
      to[short,i_=$i_{A}$](4,5)
      to[R=$R_S$](4,3.5)
      to[myl](4,2)
      to[short](3,2);
      \draw(3,6)
      to[short,i=$i_v$](6,6)
      to[R=$R_{V}$](6,4)
      to[short,i=$i_p$](5,4)
      to[R=$6.4k\ohm$](5,1)
      to[short](6,1)
      to[short](6,0)
      to[short](3,0);
      \draw(6,4)
      to[short,i=$i_l$](7,4)
      to[R=$1.6k\ohm$](7,2.5)
      to[myl](7,1)
      to[short](6,1);
      \end{circuitikz}  
    \caption{Schema aperometro a valle con generica lancetta collegata a bobina}
\end{figure}
Il circuito con l'amperometro a valle una volta collegato il circuito dell'amperometro in figura \eqref{ICE_amp} tramite la bocchetta $A_i$ e il circuito del voltmetro in figura \eqref{ICE_volt} può essere schematizzato come nella figura immediatamente sopra. Analizziamo per primo il comportamento dell'amperometro,
abbiamo $R_x$ che è la nostra resistenza da misurare, mentre le restanti nel caso $i\neq 6$ sono rispettivamente:
\begin{equation}
R_{sh} = \sum_{k=0}^{i-1}R_{i-k} \qquad R_S = R_A + \sum_{k=1}^5 R_{k+1}
\end{equation}
Nel caso $i=6$ valgono semplicemente $R_S = 720\,\ohm$,$R_{sh} = 6400\,\ohm$.
Per risolvere il circuito utilizziamo la legge di Kirchhoff sulle correnti
\begin{equation}
\label{cor}
i = i_{sh} + i_A
\end{equation}
Inoltre sappiamo dalla legge di Ohm
\begin{equation}
i_{sh} = \frac{\Delta V}{R_{sh}},\quad i_A = \frac{\Delta V}{R_S} \implies i_{sh} = i_A\frac{R_S}{R_{sh}}
\end{equation}
Se inseriamo l'equazione appena trovata nell'equazione \eqref{cor} otteniamo
\begin{equation}
\label{ia}
i=i_A\left(1 + \frac{R_S}{R_{sh}}\right) = i_A\left(\frac{R_S+R_{sh}}{R_{sh}}\right) =  i_A\left(\frac{8000}{R_{sh}}\right)
\end{equation}
La bobina mobile dello strumento che utilizziamo produce una deflessione della lancetta pari al 100\% quando è percorso da una corrente di $40\,\mu A$, quindi la corrente misurata $i_m$ con un fondo scala $FS$ vale
\begin{equation}
i_m = FS\left(\frac{i_A}{40\,\mu A}\right)
\end{equation}
Combinando questa equazione con la \eqref{ia} otteniamo
\begin{equation}
i_m = i\frac{FS\,R_{sh}}{40\mu A\,8000}
\end{equation}
Notiamo ora che il prodotto $FS\,R_{sh} = 0.32$ è sempre costante esattamente uguale a $8000*40\times 10^{-6}$. Per cui
\begin{equation}
i_m = i
\end{equation}
Ovvero la corrente misurata dall'amperometro è esattamente quella che entra all'interno dell'amperometro e che esce, quindi la corrente misurata è effettivamente la corrente che attraversa la resistenza $R_x$ che dobbiamo misurare e non occorre correggerla.\\
\\Analizziamo ora il comportamento del voltmetro, abbiamo che se collegato alla bocchetta $B_i$ con $i\neq 7$
\begin{equation}
R_V = 38740 + \sum_{k=0}^{i-1}R_{V_i}
\end{equation}
Nel caso $i=7$ abbiamo semplicemente $R_V = 720\,\ohm$. Utilizzando come prima Kirchhoff possimao scrivere
\begin{equation}
\label{iv}
i_v = i_l+i_p
\end{equation}
Dalla legge di Ohm
\begin{equation}
i_p = \frac{\Delta V'}{6400}\quad i_l = \frac{\Delta V'}{1600} \implies i_p = i_l\frac{1600}{6400}
\end{equation}
Da cui mettendo nella \eqref{iv}
\begin{equation}
i_v = i_l \left(1 + \frac{1600}{6400}\right) = 1.25i_l \implies i_l=0.8i_v
\end{equation}
Come prima la bobina mobile dello strumento che utilizziamo produce una deflessione della lancetta pari al 100\% quando è percorso da una corrente di $40\,\mu A$, quindi la tensione misurata $\Delta V_m$ con un fondo scala $FS$ vale
\begin{equation}
\label{vm}
V_m = FS\left(\frac{i_l}{40\,\mu A}\right) = FS\left(\frac{0.8i_v}{40\,\mu A}\right)
\end{equation}
Sia $\Delta V$ la differenza di potenziale tra $A$ e $B$ possiamo scrivere con la legge di Ohm e la resistenza efficace $R_{eff}$
\begin{equation}
i_v = \frac{\Delta v}{R_{eff}} = \frac{\Delta v}{R_{V}+1280}
\end{equation}
La \eqref{vm} diventa
\begin{equation}
V_m = FS\frac{0.8\Delta V}{(R_V+1280)40\,\mu A} = \Delta V\frac{FS}{R_V+1280}\frac{0.8}{40\,\mu A}
\end{equation}
Adesso basta notare che $FS/ (R_V+1280)$ e vale esattamente $0.8/(40\times 10^{-6})$ possiamo concludere quindi
\begin{equation}
\Delta V_m = \Delta V
\end{equation}
Ovvero la differenza di potenziale misurato dal voltmetro è proprio la differenza di potenziale ai suoi capi.\\
\\Però in questo circuito non si misura la d.d.p ai capi della resistenza da misurare, ma ad un capo dell'amperometro e a un capo della resistenza, si può correggere facilemente questa cosa tenendo conto che
\begin{equation}
\Delta V_m = \Delta V_A + \Delta V_R
\end{equation}
\section{Conclusioni}
\newpage
\begin{appendices}
\section{Tabelle dati}
\begin{table}[H]
\centering

  \begin{subtable}{.5\textwidth}
    \centering
    \begin{tabular}{|c|c|} \hline
     $\mathbf{i {[mA]}}$  & $\mathbf{\Delta V [V]}$  \\ \hline
      0 & 0  \\ \hline
      8.5 & 2.3  \\ \hline
      15.9 & 4.1  \\ \hline
      24 & 6.1  \\ \hline
      31.7 & 8.1  \\ \hline
      40.4 & 10  \\ \hline
      49.8 & 12.1  \\ \hline
      56.8 & 14.3  \\ \hline
      63.3 & 16  \\ \hline
      71.8 & 18.4  \\ \hline
    \end{tabular}
    \caption{Fondo scala $qualcosa$}
  \end{subtable}%
  \begin{subtable}{.5\textwidth}
  \centering
  \begin{tabular}{|c|c|} \hline
  $\mathbf{i {[mA]}}$  & $\mathbf{\Delta V [V]}$  \\ \hline
    77.6 & 20  \\ \hline
    73.4 & 18.8  \\ \hline
    66.2 & 17  \\ \hline
    59.4 & 15  \\ \hline
    52.4 & 13.1  \\ \hline
    44.2 & 11  \\ \hline
    36.2 & 8.9  \\ \hline
    26.5 & 6.9  \\ \hline
    18.9 & 5.1  \\ \hline
    -0.5 & 0  \\ \hline
  \end{tabular}
  \caption{Fondo scala $qualcosa$}
\end{subtable}
\caption{Misurazioni con circuito amperometro a monte}
\end{table}
\begin{table}[H]
\centering

  \begin{subtable}{.5\textwidth}
    \centering
    \begin{tabular}{|c|c|} \hline
     $\mathbf{i {[mA]}}$  & $\mathbf{\Delta V [V]}$  \\ \hline
      0 & 0  \\ \hline
      8.5 & 2.3  \\ \hline
      15.9 & 4.1  \\ \hline
      24 & 6.1  \\ \hline
      31.7 & 8.1  \\ \hline
      40.4 & 10  \\ \hline
      49.8 & 12.1  \\ \hline
      56.8 & 14.3  \\ \hline
      63.3 & 16  \\ \hline
      71.8 & 18.4  \\ \hline
    \end{tabular}
    \caption{Fondo scala $qualcosa$}
  \end{subtable}%
  \begin{subtable}{.5\textwidth}
  \centering
  \begin{tabular}{|c|c|} \hline
  $\mathbf{i {[mA]}}$  & $\mathbf{\Delta V [V]}$  \\ \hline
    77.6 & 20  \\ \hline
    73.4 & 18.8  \\ \hline
    66.2 & 17  \\ \hline
    59.4 & 15  \\ \hline
    52.4 & 13.1  \\ \hline
    44.2 & 11  \\ \hline
    36.2 & 8.9  \\ \hline
    26.5 & 6.9  \\ \hline
    18.9 & 5.1  \\ \hline
    -0.5 & 0  \\ \hline
  \end{tabular}
  \caption{Fondo scala $qualcosa$}
\end{subtable}
\caption{Misurazioni con circuito amperometro a valle}
\end{table}
\end{appendices}
\end{document}
