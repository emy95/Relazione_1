\documentclass[12pt,titlepage]{article}
\usepackage{mathtools}
\usepackage[italian]{babel}
\usepackage{graphicx}
\usepackage[utf8]{inputenc}
\usepackage{float}
\usepackage{tabularx}
\usepackage{chngpage}
\usepackage[toc,page]{appendix}
\usepackage{gensymb}
\usepackage{subcaption}
\usepackage{tikz}
\usepackage{circuitikz}

\addtolength{\topmargin}{-.875in}
\textheight=674pt

\title{\textbf{MISURE VOLT- AMPEROMETRICHE} }
\author{Canteri Marco\\Biasi Lorenzo\\Damiani Emily}
\date{}

\begin{document}
	\maketitle
	\tableofcontents
	\renewcommand{\abstractname}{Abstract}
	
	\begin{abstract}
In questa esperienza misureremo il valore di una resistenza analizzando con un tester ICE la differenze di potenziale ai suoi capi ela corrente che gli passa attraverso. Verificheremo quindi il comportamento ohmico di tale resistenza. Infine studiaremo la caratteristica Volt-Amperometrica di una lampadina.
	\end{abstract}

\section{Obiettivi}
\begin{itemize}
\item Determinare il valore di resistenze tramite misure volt-amperometriche con il tester
ICE, in configurazione amperometro “a monte” e “a valle”.
\item Verificare il comportamento ohmico della resistenza e confrontare i valori ottenuti e
le loro incertezze tra di loro e con il valore ottenuto da un multimetro digitale. 
\item Graficare e commentare la caratteristica Volt-Amperometrica di una lampadina
\end{itemize}
\section{Strumenti}
\section{Procedura di misura}
\section{Circuito amperometro a monte}
\section{Cirucito amperometro a valle}
\section{Analisi circuito tester ICE}
\begin{figure}[h!]
  \centering
    \begin{circuitikz}
      \draw (0,0) 
       node[label={below:A}] {}
       node[yshift=-1ex] {=}
       to[short,o-*] (0,2)
       to[short] (0,4)
       to[short] (5.5,4)
       to[ammeter]node[right = -0.5em,yshift=-2.3ex] {$R_A = 1600\ohm$} node[right = -7.5em,yshift=-2.3ex] {$I_{fs} = 40\mu A$} 
       node[right = 0.1em,yshift=2.3ex] {$+$} node[right =-3.5em,yshift=2.3ex] {$-$}(6.5,4)
       to[short] (12,4)
       to[short,-*] (12,2)
       to[R= $720\ohm$,-o](12,0)node[right = 0.1em] {$A_6$}
       node[label={below:$50\mu A$}] {};
      \draw (0,2)
      to[R=$R_1$,*-*]node[below,right=-3.6em,yshift=-2.3ex] {0.064\ohm}(2,2)
      to[R=$R_2$,*-*]node[below,right=-3.6em,yshift=-2.3ex] {0.576\ohm}(4,2)
      to[R=$R_3$,*-*]node[below,right=-3.4em,yshift=-2.3ex] {5.76\ohm}(6,2)
      to[R=$R_4$,*-*]node[below,right=-3.4em,yshift=-2.3ex] {57.6\ohm}(8,2)
      to[R=$R_5$,*-*]node[below,right=-3.4em,yshift=-2.3ex] {576\ohm}(10,2)
      to[R=$R_6$,*-*]node[below,right=-3.6em,yshift=-2.3ex] {5760\ohm}(12,2);
      \draw (2,2)
      to[short,-o](2,0)node[right = 0.1em] {$A_1$}
      node[label={below:5A}] {};
      \draw (4,2)
      to[short,-o](4,0)node[right = 0.1em] {$A_2$}
      node[label={below:500mA}] {};
      \draw (6,2)
      to[short,-o](6,0)node[right = 0.1em] {$A_3$}
      node[label={below:50mA}] {};
      \draw (8,2)
      to[short,-o](8,0)node[right = 0.1em] {$A_4$}
      node[label={below:5mA}] {};
      \draw (10,2)
      to[short,-o](10,0)node[right = 0.1em] {$A_5$}
      node[label={below:$500\mu A$}] {};
    \end{circuitikz}
    \caption{Schema semplificato tester ICE amperometro}
    \label{ICE_amp}
\end{figure}
Il circuito in figura \eqref{ICE_amp} una volta collegato all'interno del circuito tramite la bocchetta $A_i$ può essere schemattizato come segue
\begin{figure}[H]
  \centering
    \begin{circuitikz}
      \draw (0,0) 
      %node[label={below:$A_i$}] {}
      node[xshift=-1.5ex] {$A_i$}
      to[short,o-*](2,0)
      to[R=$R_{sh}$,-*](2,-2)
      to[short,-o]node[xshift=-4.0ex] {A}(0,-2);
      \draw(2,0)
      to[R=$R_S$](4,0)
      to[ammeter](6,0)
      to[R=$R_A$](6,-2)
      to[short,-*](2,-2);
      \end{circuitikz}  
\end{figure}
\section{Conclusioni}
\end{document}
